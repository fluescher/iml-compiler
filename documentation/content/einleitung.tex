Es ist Wahljahr. Die Wahlkampagnen der Parteien beginnen langsam in Fahrt zu kommen. Dabei fällt insbesondere die SVP auf. Trotz der enormen Präsenz
der Energieproblematik in den Medien und der intensiven Auseinandersetzung der Bevölkerung damit, versucht die SVP  
ihr Hauptthema wieder in den Fokus der Öffentlichkeit zu bringen; die Zuwanderung.
Insbesondere zielt sie auf das Personenfreizügigkeitsabkommen zwischen der Schweiz und der EU. Dieses müsse neu verhandelt werden,
damit die Schweiz die Kontrolle über die Zuwanderung zurückgewinnen könne. Die anderen politischen 
Parteien sowie die Wirtschaftsverbände begehren jedoch auf, da für sie die Personenfreizügigkeit ein zentraler Baustein des
wirtschaftlichen Erfolgs der Schweiz ist und unter keinen Umständen gefährdet werden dürfe.
Trotzdem wird die SVP damit Wahlkampf führen und begründet dies unter anderem damit, dass die Auswirkungen der Personenfreizügigkeit auf die Sozialwerke gravierend seien \cite{SVP:Wahlplattform11}.
Doch welche Auswirkungen hat dieses Abkommen tatsächlich auf die schweizerischen Sozialwerke?
