%% Einfluss auf die ALV

Die Zuwanderung hat aber nicht nur  direkten Einfluss auf die Konjunktur und die Arbeitsmarktentwicklung, sondern auch auf die Arbeitslosenversicherung (ALV). 
Die Gegner der Personenfreizügigkeit befürchten, dass die zugewanderten Arbeitskräfte während eines 
wirtschaftlichen Aufschwungs in die Schweiz ziehen, jedoch in einer Krise die Sozialversicherungen 
auserordentlich stark belasten würden.
Nun, da die Schweiz das erste Mal seit der Einführung der Personenfreizügigkeit eine globale Krise 
durchlebt, kann die reale Situation das erste Mal klar umrissen werden. Im Krisenjahr 2009 
haben dabei tatsächlich EU Staatsangehörige mehr Arbeitslosenentschädigunt bezogen als eingezahlt 
(21\% der Einzahlungen gegenüber 23\% der Beiträge). Während einer Krise entstehen
also für die ALV Mehrkosten. Diese werden sich für das Jahr 2010 auf 115 Millionen Franken belaufen \cite[S. 8]{ADMIN:Bericht}.
Diese Mehrkosten entstanden dadurch, dass neu auch Saisonarbeitskräfte mit Kurzaufenthaltsbewilligungen 
Arbeitslosengeld beziehen dürfen, falls sie ALV-Beiträge eingezahlt haben \cite[S. 46]{ADMIN:Auswirkungen}.
Es muss also durchaus davon ausgegangen werden, dass die Zuwanderung die Arbeitslosenversicherung auch in 
Zukunft während einer Krise stärker belasten wird.
