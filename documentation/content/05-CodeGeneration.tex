\section{Code Generation}

Die Zielplattform des generierten Codes ist die Java Virtual Machine \cite{ORACLE:JVM}. Daher
wird Java Bytecode generiert. Dies kann mit den selben Prinzipien angegangen werden, welche auch
bei der Generierung von Maschinen Code oder Code für die im Unterricht zur Verfügung gestellte 
VM gelten. Da die JVM jedoch von Beginn an auf die Sprache Java zugeschnitten wurde, gibt es bei
der Abbildung eines IML Programmes auf die Struktur eines JVM Programmes einige Spezialitäten, 
auf welche in diesem Abschnitt eingegangen wird. Für das Schreiben der \textit{.class} Dateien wird
die verbreitete Bibliothek ASM \cite{OW2:ASM} verwendet.

\subsection {Strukturierung}

Die JVM ist wie die Sprache Java objektorientiert aufgebaut. Aus diesem Grund muss ein 
IML Programm auf die Klassenstruktur der JVM abgebildet werden. In diesem Abschnitt wird 
anhand des Beispielprogramms aus Listing \ref{lst:outtest} die Abbildung auf eine JVM Klasse
erläutert. Dieses IML Programm wird zu einer Klasse, welcher der Klassendeklaration in Listing 
\ref{lst:outtest_class} entspricht. Dabei wurde folgende Abbildung gewählt:

\begin{itemize}
    \item \textbf{Programm}: Ein IML Programm wird auf eine Klasse mit gleichem Namen abgebildet.
                            Um die Kompatibilität mit andern JVM Sprachen zu gewährleisten werden die globalen
                            Variablen im Konstruktor mit Defaultwerten initialisiert. Die Klasse wurde als final 
                            gekennzeichnet, um späteres überschreiben von Methoden zu verhindern.
    \item \textbf{Globale Variablen}: Globale Variablen werden zu privaten Feldern der Klasse des Programms.
    \item \textbf{Programm Body}: Die Commands des IML Programms werden zu einer neuen Funktion der Klasse,
                                 welche den Namen der Klasse hat, kompiliert. Der Vorteil im Gegensatz zum 
                                 direkten kompilieren in die main Methode liegt darin, dass die globalen Variablen 
                                 und Routinen nicht \textit{static} sein müssen und so mehrere Instanzen des 
                                 IML Programms in der selben VM aktiv sein können, ohne Zustand teilen zu müssen. 
                                 Dies würde es auch erlauben Bibliotheksfunktionen in IML zu entwickeln und 
                                 in anderen JVM Sprachen zu verwenden.
    \item \textbf{Routinen}: Routinen werden zu Funktionen der Klasse. Wenn keine out-Parameter verwendet
                             werden entsprechen diese Funktionen normalen JVM Funktionen und stehen ohne 
                             spezielle Konventionen an den Aufruf zur Verfügung. Wie out-Parameter abgebildet 
                             werden ist im nächsten Abschnitt erläutert.
    \item \textbf{Eintrittpunkt}: Falls die JVM eine Klasse ausführen will, so ruft sie die Funktion \textbf{main}
                                  auf. Daher wird ebenfalls eine solche generiert, welche eine neue Instanz der 
                                  Klasse erzeugt und die Methode mit dem gleichen Namen der Klasse ausführt.
\end{itemize}
\vspace{0.5cm}
\begin{lstlisting}[caption=Laden eines Int-Wertes in zwei Speicherstellen über out Parameter,label={lst:outtest}]
program outtest
global
    proc loadTwice(ref i: int32, out ref a: int32, out ref b: int32)
    requires[ i > 0 : exampleCondition ]
    ensures[ a = old(i), b = old(i) ]
    {
        a init := i;
        b init := i
    };
    a: int32;
    b: int32;
    var v: int32
{
    v init := 12;
    call loadTwice(v, a init, b init);
    ! a;
    ! b
}
\end{lstlisting}

\begin{lstlisting}[caption=Struktur der erzeugten Java-Klasse,language=Java,label={lst:outtest_class}]
public final class outtest {
    public void loadTwice(int, int[], int[]);
    public outtest();
    public void outtest();
    public static void main(java.lang.String[]);
}
\end{lstlisting}


\subsection {Parameterübergabe}

IML unterstützt ein sehr ausgefeiltes Handling von Parameterwerten. Die JVM jedoch erlaubt 
das Übergeben von Parametern lediglich \textit{by-value}. Zudem ist es in IML möglich 
festzulegen, ob ein Parameter ein Eingabeparameter, Ausgabeparameter oder beides ist. Auch hier 
erlaubt die JVM lediglich Eingabeparameter. Um Parameter von IML vollständig abbilden zu können 
werden auch \textit{ref} Parameter wie \textit{copy} Parameter behandelt. \textit{out}-Parameter 
können so jedoch nicht realisiert werden, da die JVM dafür Referenzen auf primitive Typen 
unterstützen müsste. Um nun das Schreiben von Informationen in den Speicherbereich des Aufrufers
zu erlauben, könnte eine Hilfsklasse (z.B. IntRef) erzeugt werden. Dies würde jedoch eine 
Abhängigkeit zu dieser Klasse für den generierten Code und jeden, der diesen Benutzen möchte, 
zur Folge haben. Dies kann umgangen werden, wenn der entsprechende Parameter in ein Array mit
der Länge 1 umgewandelt wird. So können \textit{out} Parameter auf der JVM elegant implementiert werden.
Dies bedeutet für den Aufrufer jedoch mehr Aufwand, da er zunächst für jeden \textit{out} Parameter 
ein neues Array erzeugen und den zu übergebenden Wert(im Falle von \textit{inout}) hinein kopieren muss. 
Im Anschluss an den Methodenaufruf muss der Wert im Array wieder an die Speicherstelle des ursprünglichen 
Wertes kopiert werden. Für den Aufgerufenen fällt jedoch keinerlei Zusatzaufwand an, wie in Listing 
\ref{lst:loadtwice_code} zu sehen ist.
\newline

\begin{lstlisting}[caption=Bytecode der loadTwice Prozedur ohne conditions,label={lst:loadtwice_code}]
public void loadTwice(int, int[], int[]);
    flags: ACC_PUBLIC
    Code:
      stack=3, locals=4, args_size=4
        0: aload_2          // Laden des 2. Parameters auf Stack (Array Referenz)
        1: iconst_0         // Laden von 0 auf den Stack (Pos in Array)
        2: iload_1          // Laden des ersten Parameters auf Stack (zu speichernder Wert)
        3: iastore          // Speichern (int array store)
        4: aload_3       
        5: iconst_0      
        6: iload_1       
        7: iastore       
        8: return
\end{lstlisting}


\subsection{Pre-/Postconditions}

Für das Umsetzen der Pre-/Postconditions kann gewöhnlicher Code generiert werden. Falls eine der Conditions
jedoch fehlschlägt, muss die Programmausführung unterbrochen werden. Da Java bereits ein Konstrukt für 
Assertions unterstützt, gibt es bereits in der Standartbibliothek die AssertionError Exception.
Diese kann auch für die Conditions in IML verwendet werden, wobei das Label der Condition, falls vorhanden,
als Exception Message verwendet wird.
\newline

\begin{lstlisting}[caption=Bytecode der loadTwice Prozedur mit precondition,label={lst:loadtwice_code_precond}]
public void loadTwice(int, int[], int[]);
    flags: ACC_PUBLIC
    Code:
      stack=3, locals=4, args_size=4
         0: iload_1       
         1: bipush        0
         3: if_icmple     10
         6: iconst_1      
         7: goto          11
        10: iconst_0      
        11: ifeq          17
        14: goto          27              // Weiter mit normalem Code
        17: new           #9              // class java/lang/AssertionError
        20: dup         
        21: ldc           #11             // Laden der String Konstante "exampleCondition"
        23: invokespecial #15             // Method java/lang/AssertionError."<init>"
        26: athrow                        // Werfen der Exception
        27: aload_2       
        28: iconst_0      
        29: iload_1       
        30: iastore       
        31: aload_3       
        32: iconst_0      
        33: iload_1       
        34: iastore       
        35: return  
\end{lstlisting}

\subsubsection{Speichern des Preexecution States}

Der Preexecution State wird im Stack Frame der Methode als lokale Variable gespeichert. Wenn nun 
die Postcondtitions ausgewertet werden, wird im Falle eines Calls der \textit{old} Funktion der Wert 
an der entsprechenden Position der lokalen Variable geliefert.
\newline
\begin{lstlisting}[caption=Bytecode der loadTwice Prozedur mit postcondition mit Zugriff auf old State. Damit der Code etwas Übersichtlicher ist\, wurde nur die erste Postcondtition kompiliert.,label={lst:loadtwice_code_old}]
public void loadTwice(int, int[], int[]);
    flags: ACC_PUBLIC
    Code:
      stack=3, locals=5, args_size=4
         0: iload_1                 // Expression in old(i) -> i 
         1: istore        4         // Speichern an lokalem Variablen Index 4
         3: aload_2                 // Normaler Code
         4: iconst_0      
         5: iload_1       
         6: iastore       
         7: aload_3       
         8: iconst_0      
         9: iload_1       
        10: iastore       
        11: aload_2       
        12: iconst_0      
        13: iaload        
        14: iload         4         // Zuvor gespeicherter Wert laden
        16: if_icmpne     23
        19: iconst_1      
        20: goto          24
        23: iconst_0      
        24: ifeq          30
        27: goto          38
        30: new           #9        // class java/lang/AssertionError
        33: dup           
        34: invokespecial #13       // Method java/lang/AssertionError."<init>":()V
        37: athrow        
        38: return
\end{lstlisting}


