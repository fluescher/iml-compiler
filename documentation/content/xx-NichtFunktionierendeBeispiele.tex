
\begin{lstlisting}[caption=Wert einer Variable in der pre-/postcondition ändern]
proc divide(in copy m:int32, in copy n:int32, out ref q:int, out ref r:int32)
requires [n := 0]
ensures [r := 12 + 1]
{
    q init := 0;
    r init := m;

    while(r >= n) {
        q := q+1;
        r := r-n;
    }
}
\end{lstlisting}

\begin{lstlisting}[caption=Zugriff auf eine Variable mit der old Funktion, welche nicht Vorhanden ist.]
fun gcd(in copy a:int32, in copy b:int32) returns r:int32
requires [a > 0, b > 0]
ensures [old(x) > 0]
{
    r init := 0;
    
    if (a == 0) {
        r := b;
    } else {
        while (b /= 0) {
            if (a > b) {
                a := a - b;
            } else {
                b := b - a;
            }
        }
    }

    r := a
}
\end{lstlisting}

\begin{lstlisting}[caption=Eine nicht Boolsche Expression in der Condition List]
fun multiply(in copy m:int32, in copy n:int32) returns r:int32
requires [n + 0]
{
    i init := 0;
    r init := m;

    while(i < n) {
        i := i+1;
        r := r+m;
    }
}
\end{lstlisting}

\begin{lstlisting}[caption=Eine Funktion in der Condition List welche keinen Boolschen Wert zurückliefert]
fun multiply(in copy m:int32, in copy n:int32) returns r:int32
requires [positive:notBool(n)]
{
    i init := 0;
    r init := m;

    while(i < n) {
        i := i+1;
        r := r+m;
    }
}
fun notBool(in copy n:int32) returns r:int32
{
    r init := 10;
    if(n>0) {
        r := 5;
    } else {
        r := 7;
    }
}
\end{lstlisting}

\begin{lstlisting}[caption=????]
proc twoTimes(in copy m:int32, out ref r:int32)
ensures [old(m) = (r - m)]
{
    r init := m + m;
}
\end{lstlisting}
