\section{Kontext- und Typeinschränkungen}
\label{sec:constraints}
In diesem Kapitel werden die nötigen Kontext Checks beschrieben.

\subsection{Kontext Einschränkungen}

Für conditions gilt grundsätzlich, dass bei ihrer Ausführung kein State verändert werden 
darf. In IML wurde dies durch die saubere Trennung von Expressions und Commands, sowie die standard
Kontext Checks bereits erreicht. Trotzdem sind einige weitere Einschränkungen nötig.

\begin{itemize}

\item Die Funktion old() darf nur innerhalb von postconditions aufgerufen werden.
\item Es darf keine weitere Funktion mit dem Namen old deklariert werden.
\item Das Label einer Condition muss innerhalb der Condition List einmalig sein.
\item Expressions innerhalb der Conditions müssen einen boolschen Wert erzeugen.
\item In Conditions darf kein Rekursiver Aufruf an die Routine zu der sie gehört stattfinden.

\end{itemize}

\subsubsection{Speicherzugriff}

\begin{itemize}
\item In Preconditions kann auf alle initialisierten Variablen und Konstanten zugegriffen werden, welche 
in der Parameterliste oder der Global Import List definiert wurden. Auf lokal deklarierte Variablen 
kann nicht zugegriffen werden. Da out-Parameter nicht initialisiert sind, stehen sie nicht zur
Verfügung.
\item In Postconditions kann auf alle Variablen zugegriffen werden, welche auch in 
Preconditions möglich sind. 
\item In Postconditions von Funktionen kann auf die Returnvariable zugegriffen werden.
\item In Postconditions von Prozeduren kann auf out-Parameter zugegriffen werden.
\item In einer Postcondition kann mit old() eine Expression im preexecution State ausgeführt werden. Für diese Expression 
gelten daher auch die Einschränkungen des preexecution state.

\end{itemize}

%\newpage

%\begin{figure*}[h]
%	\begin{center}
%		\includegraphics[width=0.9\textwidth]{images/Zuwanderungssaldo_Bericht_2.png}
%	\end{center}
%	\caption{Verlauf der Zuwanderung nach Herkunftsländern in Tausend \cite[S. 18]{ADMIN:Bericht}}
%	\label{fig:zuwanderungsaldi}
%\end{figure*}

