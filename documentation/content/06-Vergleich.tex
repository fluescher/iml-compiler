\section{Vergleich mit anderen Programmiersprachen}
In diesem Abschnitt wird der gewählte Entwurf mit Implementierungen
für andere Sprachen verglichen.

\subsection{Java / C(++)}
Die meist verwendeten Sprachen Java, sowie C(++) enthalten keine Unterstützung des Compilers 
zur Definition von pre-/postconditions auf Methoden. Jedoch gibt es die Möglichkeit 
über \textit{Assertions} Bedingungen im Code zu definieren, welche zu einem Programmabbruch führen.
Darüber hat ein Programmierer die Möglichkeit conditions zu formulieren. Diese können jedoch über weite
Teile des Quellcodes verteilt sein und es ist für einen anderen Programmierer nicht 
sofort ersichtlich welche pre-/postconditions daraus abgeleitet werden können. Im speziellen 
Fall von Java ist es zudem so, dass die Ausführung von Assertions defaultmässig nicht aktiviert 
ist. Daher ist der Nutzen solcher Assertions leider nur gering. Da diese Sprachen jedoch 
sehr weit verbreitet sind haben sich zahlreiche Frameworks entwickelt, welche es erlauben
zusätzliche Bedingungen an Methodenaufrufe zu knüpfen. Diese sind jedoch nicht im Compiler integriert
und haben sich daher auch nicht weit verbreitet.

\subsection{C\# (.NET Framework)}

Für die Common Language Runtime (CLR) von Microsoft gilt eigentlich dasselbe wie für die Sprachen 
Java und C(++). Es muss jedoch erwähnt werden, dass von Microsoft Research Tools entwickelt wurden, 
die in dieser Form in keiner der obengennanten Sprachen umgesetzt wurden. Das Framework 
\textit{Code Contracts} ermöglicht es dem Programmierer über statische Methodenaufrufe 
conditions(oder eben contracts) zu formulieren (ähnlich wie Assertions). 
Danach ist es dem Framework möglich, daraus sowohl Laufzeitchecks als auch 
Dokumentationsdateien zu generieren. Viel interessanter ist jedoch die Möglichkeit einen 
statischen Checker zu benutzen, welcher Contracts bereits zur compile-time überprüfen 
kann \cite{MS:StaticAnalysis}. Dies geht viel weiter als unsere Implementirung für IML, 
ist jedoch nicht direkt in der Sprache enthalten.

\subsection{Eiffel}
Die Sprache Eiffel enthält sehr ähnliche Implementierung wie die von uns für IML gewählte. Durch
die objektorientierte Natur von Eiffel unterstützt sie zusätzlich Klasseninvarianten, wie auch die
Vererbung von conditions. Den Zugriff auf pre-execution State von Variablen wir über das 
Keyword \textit{old} ermöglicht.

\subsection{D}
Die Sprache D\cite{D:Main}, welche C++ weiterentwickeln soll, ermöglicht es sowohl Klassen-Invarianten,
als auch conditions auf der Ebene eines Statements zu definieren. Da auch diese Sprache objektorientiert 
ist, wurden Klasseninvarianten implementiert. Es ist jedoch nicht möglich innerhalb der Postcondition 
auf den Zustand von Variablen vor der Ausführung des bodys zuzugreifen.\newline

\begin{lstlisting}[caption=Beispiel in D]
in 
{ 
    assert(x >= 0);
} 
out (result) 
{
    assert((result * result) <= x && (result+1) * (result+1) >= x);
} 
body
{
    ... code ...
}
\end{lstlisting}

\subsection{Ada}
Ada erlaubt die Definition von pre-postconditions auf Ebene von Prozeduren/Funktionen und Subtypen 
seit der Version 2012. Subtypen sind
dabei bereits eine Art von Einschränkung, welche vom Compiler als Typ aufgefasst werden und daher bereits
bei der Kompilationszeit erkannt werden. Seit Ada 2012 ist es jedoch auch möglich, solche Subtypen 
mit dynamischen und statischen Prädikaten zu versehen. Dabei werden die dynamischen 
überprüft, während die statischen viele Überprüfungen bereits zur Kompilationszeit
 durchführen können. Auch 
Ada erlaubt den Zugriff auf Zustand vor der Ausführung der Prozedur. \newline

\begin{lstlisting}[caption=Beispiel in Ada 2012,language=Ada]
procedure Update_Person (P : in out Person)
   with Post => P.Sex = P.Sex'Old 
                and P.Birth_Date = P.Birth_Date'Old;

function Inc(X: Integer) return Integer
   with Pre  => X /= Integer'Last,
        Post => Inc'Result = X'Old+1;
\end{lstlisting}
