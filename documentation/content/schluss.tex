Wie bei allem gibt es auch bei der Personenfreizügigkeit Vor- und Nachteile.
Auf der einen Seite belebt die Zuwanderung die Konjunktur, andererseits 
wird durch die gewachsene Bevölkerung die Schweizer Infrastruktur weiter belastet. Entgegen der Darstellung der SVP,
hat die Zuwanderung jedoch kaum negative Einflüsse auf die Sozialversicherungen. Tatsächlich leistet sie einen wichtigen Beitrag
zur Finanzierung der AHV und der IV. Das Anliegen der SVP, die Personenfreizügigkeit neu zu verhandeln würde den wirtschaftlichen 
Erfolg der Schweiz gefährden und die Sozialversicherungen unnötig belasten. Dennoch können die zukünftigen Herausforderungen
im Zusammenhang der Zuwanderung nicht ignoriert werden. Um diese anzugehen braucht es ehrliche Lösungen in den drängensten 
Bereichen der bereits jetzt ausgelasteten Infrastruktur und der Raumplanung. Nur so kann die Schweiz auch in Zukunft 
eine wettbewerbsfähige Volkswirtschaft bleiben und von der Zuwanderung profitieren.
