Obwohl in den vergangenen Jahrzehnten im Bereich des Softwareengineerings viel 
Aufwand in das Entwickeln von Technologien zur effizienteren Entwicklung robuster
Software gesteckt wurde, gibt es noch immer wenige Programmiersprachen welche 
das Schreiben robuster Software beim Sprach-Design berücksichtigt haben. Eine der Ausnahmen 
ist die Sprache Eiffel, welche beispielsweise das Definieren von Pre-/Postconditions
ermöglicht. Jedoch gibt es vermehrt auch Frameworks für etabliertere Sprachen, welche 
es erlauben, Elemente des Design-by-Contract anzuwenden und vom Compiler zu überprüfen.
In der Sprache C\# von Microsoft beispielsweise wird dies mittels \textit{Code Contracts}
 ermöglicht, welche auch vom Compiler statisch analysiert werden \cite{MS:CodeContracts}.
Auf der Java Platform gibt es über den Community Process Anstrengungen, über Annotationen
zusätzliche Metainformationen bereitszustellen, welche von einem externen Tool (z.B. FindBugs) statisch analysiert 
werden können und den Programmierer bei möglichen Fehlern warnen \cite{JSR:305}. Jedoch wurde dies
auch im aktuellen Compiler der Java Version 7 noch nicht implementiert. In diesem Bericht werden Erweiterungen
der Unterrichtsprache IML beschrieben, welche durch das Einführen von Pre-/Postconditions das Schreiben robuster Software erleichtern sollen.

